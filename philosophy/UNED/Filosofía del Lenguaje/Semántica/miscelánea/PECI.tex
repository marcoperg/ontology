\documentclass[12pt,a4paper]{article}
\usepackage[spanish]{babel}
\usepackage[utf8]{inputenc}
\usepackage[T1]{fontenc}
\usepackage{setspace}
\usepackage{geometry}
\geometry{margin=2.5cm}
\setlength{\parskip}{1em}  % You can adjust '1em' as needed


\begin{document}

\begin{center}
    \Large \textbf{Filosofía del lenguaje I}\\[0.3cm]
    \large PEC I (Temas 1--3)\\[0.5cm]
    \normalsize \textbf{Nombre y apellidos:} \underline{Marco Pérez González}

\end{center}

\vspace{1cm}

\noindent
\textbf{Instrucciones:} Conteste a dos de las siguientes preguntas. Extensión máxima por pregunta: 300 palabras.

\vspace{1cm}

\noindent\textbf{1.} Los enunciados de identidad informativos entre términos co-referenciales suponen un problema para las teorías que entienden el significado como referencia. ¿Por qué? ¿Cómo explicaría Frege el valor cognoscitivo o informativo del enunciado de identidad ``Leopoldo Alas es Clarín''?

\vspace{1cm}

Los enunciados de identidad informativos son aquellos que relacionan dos términos no idénticos ---i.e. cuyos signos son distintos--- de forma tal que el enunciado es cierto cuando estos dos términos se refieren a la misma cosa. En una teoría que entiende el significado como referencia, la igualdad entre términos se establecerá de forma extensional. Es decir, como el significado de dos términos no es otra cosa que aquello a lo que refieren, dos términos serán iguales cuando, y sólo cuando, se refieran a la misma cosa. Como en estas teorías la relación de identidad se establece ignorando el sentido no tenemos forma de diferenciar entre enunciados de identidad informativa y no informativa.

Veamos un ejemplo, el enunciado $2=2$ y $1+1=2$ tienen el mismo valor de verdad y de información en una teoría extensional de la verdad, ya que ambas establecen igualdad entre términos con la misma referencia, el número $2$. Sin embargo, parece que el segundo nos aparta más información que el primero, he aquí el problema principal de este tipo de teorías.

Frege intenta solucionar este problema basándose en su distinción entre sentido y referencia. En el enunciado ''Leopoldo Alas es Clarín'' se postula la identidad de dos términos co-referenciales pero con sentidos distintos. De ahí se extraería el valor informativo del enunciado, de la afirmación que dos términos con sentidos distintos coinciden en referencia. De esta forma, alguien que fuese lector de los artículos de Clarín podría conocer en persona a un tal Leopoldo Alas sin conocer en primera instancia que eran la misma persona. Para este hipotético individuo el sentido de Clarín sería el del escritor de aquellos artículos, y el sentido de Leopoldo Alas el de aquel hombre que conocí ese día. Y el valor cognoscitivo del enunciado a analizar sería precisamente el de entender que ambos nombres hablan en realidad de la misma persona.


\vspace{2cm}

\noindent\textbf{2.} Enuncie el principio de composicionalidad del significado e ilústrelo mediante algún ejemplo.

\vspace{2cm}

\noindent\textbf{3.} ¿Cómo analizaría Russell el enunciado ``El autor de \textit{El Quijote} es peruano''?

\vspace{2cm}

\noindent\textbf{4.} ¿Qué es una proposición singular o russelliana y cómo se distingue de una proposición fregeana? Presente dos enunciados que expresen dos proposiciones fregeanas diferentes pero la misma proposición russelliana.


\vspace{1cm}

Análicemos los dos siguientes enunciados:
\begin{itemize}
\item Aristóteles nació en Estagira.
\item El maestro de Alejandro Magno nació en Estagira.
\end{itemize}

\end{document}


%%% Local Variables:
%%% mode: LaTeX
%%% TeX-master: t
%%% End:
