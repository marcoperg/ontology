\documentclass[12pt,a4paper]{article}
\usepackage[spanish]{babel}
\usepackage[utf8]{inputenc}
\usepackage[T1]{fontenc}
\usepackage{setspace}
\usepackage{geometry}
\geometry{margin=2.5cm}

\begin{document}

\begin{center}
    \Large \textbf{Filosofía del lenguaje I}\\[0.3cm]
    \large PEC I (Temas 1--3)\\[0.5cm]
    \normalsize \textbf{Nombre y apellidos:} \rule{10cm}{0.4pt}
\end{center}

\vspace{1cm}

\noindent
\textbf{Instrucciones:} Conteste a dos de las siguientes preguntas. Extensión máxima por pregunta: 300 palabras.

\vspace{1cm}

\noindent\textbf{1.} Los enunciados de identidad informativos entre términos co-referenciales suponen un problema para las teorías que entienden el significado como referencia. ¿Por qué? ¿Cómo explicaría Frege el valor cognoscitivo o informativo del enunciado de identidad ``Leopoldo Alas es Clarín''?

\vspace{1cm}

Problema de la igualdad extensional.

\vspace{2cm}

\noindent\textbf{2.} Enuncie el principio de composicionalidad del significado e ilústrelo mediante algún ejemplo.

\vspace{2cm}

\noindent\textbf{3.} ¿Cómo analizaría Russell el enunciado ``El autor de \textit{El Quijote} es peruano''?

\vspace{2cm}

\noindent\textbf{4.} ¿Qué es una proposición singular o russelliana y cómo se distingue de una proposición fregeana? Presente dos enunciados que expresen dos proposiciones fregeanas diferentes pero la misma proposición russelliana.

\vspace{2cm}

\end{document}


%%% Local Variables:
%%% mode: LaTeX
%%% TeX-master: t
%%% End:
