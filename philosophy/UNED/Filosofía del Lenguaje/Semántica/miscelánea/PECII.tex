\documentclass[12pt,a4paper]{article}
\usepackage[spanish]{babel}
\usepackage[utf8]{inputenc}
\usepackage[T1]{fontenc}
\usepackage{setspace}
\usepackage{geometry}
\geometry{margin=2.5cm}

\begin{document}

\begin{center}
    \Large \textbf{Filosofía del lenguaje I}\\[0.3cm]
    \large PEC II (Temas 4--6)\\[0.5cm]
    \normalsize \textbf{Nombre y apellidos:} \rule{10cm}{0.4pt}
\end{center}

\vspace{1cm}

\noindent
\textbf{Instrucciones:} Conteste a dos de las siguientes preguntas. Extensión máxima por pregunta: 300 palabras.

\vspace{1cm}

\noindent\textbf{1.} Kripke ofrece varios argumentos en contra de las teorías descriptivistas. Explique uno de ellos.

\vspace{2cm}

\noindent\textbf{2.} ¿Cuál es, según Kripke en \textit{El nombrar y la necesidad}, el estatus epistemológico y metafísico de ``La barra B tiene un metro de longitud en t0''?

\vspace{2cm}

\noindent\textbf{3.} Putnam sostiene que, si los significados ``están en la cabeza'', o son equivalentes a estados psicológicos, entonces no determinan la referencia. ¿Qué experimento mental usa para argumentar en favor de esta tesis? ¿Cómo se denomina su posición?

\vspace{2cm}

\noindent\textbf{4.} ¿Qué tipo de teoría es adecuada como teoría del significado según Davidson?

\vspace{2cm}

\end{document}

%%% Local Variables:
%%% mode: LaTeX
%%% TeX-master: t
%%% End:
