\documentclass[12pt,a4paper]{article}
\usepackage[spanish]{babel}
\usepackage[utf8]{inputenc}
\usepackage[T1]{fontenc}
\usepackage{setspace}
\usepackage{geometry}
\geometry{margin=2.5cm}

\begin{document}

\begin{center}
    \Large \textbf{Filosofía del lenguaje I}\\[0.3cm]
    \large PEC I (Temas 1--3)\\[0.5cm]
    \normalsize \textbf{Nombre y apellidos:} \underline{Marco Pérez González}

\end{center}

\vspace{1cm}

\noindent
\textbf{Instrucciones:} Conteste a dos de las siguientes preguntas. Extensión máxima por pregunta: 300 palabras.

\vspace{1cm}

\noindent\textbf{1.} Kripke ofrece varios argumentos en contra de las teorías descriptivistas. Explique uno de ellos.

\vspace{2cm}

\noindent\textbf{2.} ¿Cuál es, según Kripke en \textit{El nombrar y la necesidad}, el estatus epistemológico y metafísico de ``La barra $B$ tiene un metro de longitud en $t_0$''?

\vspace{1cm}

En la primera conferencia de \textit{El nombrar y la necesidad}, Kripke distingue entre las verdades \textit{a priori}, aquellas que pueden conocerse independientemente de cualquier experiencia; y las verdades necesarias, que son aquellas que son ciertas en todos los mundos posibles. La noción de aprioridad es una noción epistemológica, ya que nos habla de las características de conocimiento de un enunciado, mientras que la noción de necesidad es puramente metafísica.

Si definimos el metro como la longitud de una determinada barra $B$ en un momento concreto $t_0$, la afirmación ``La barra $B$ tiene un metro de longitud en $t_0$'' es, según Kripke, una verdad \textit{a priori} pero no necesaria. Epistemológicamente es \textit{a priori} ya que un hablante $A$ que define el metro de la forma anterior no necesitará interacción alguna con el mundo para determianr que $B$ mide, efectivamente, un metro. Sin embargo, esta misma barra podría haber medido algo distinto si se le hubiese sometido, en algún momento anterior, a una temperatura o a un esfuerzo distinto. Kripke explica que el metro es un designador rígido que refiere a una longitud determinada, que nuestro hablante ha definidio en función de la barra $B$ pero que se podría haber definido usando, como se hace hoy en día, la longitud que recorre la luz en un determinado tiempo. Por lo tanto, el hecho de que la barra mide lo que mide ---esto es, un metro, lo que mide en este mundo esta barra $B$ en $t_0$ o la distancia que recorre la luz en un tiempo concreto-- en este momento es un hecho contingente, podría no haber sido así. Esto no contradice la aprioridad del enunciado, dada la definición que hace $A$ del metro, se tiene inmediatamente y de forma independiente de toda experiencia que $B$ mide un metro en $t_0$.

\vspace{2cm}

\noindent\textbf{3.} Putnam sostiene que, si los significados ``están en la cabeza'', o son equivalentes a estados psicológicos, entonces no determinan la referencia. ¿Qué experimento mental usa para argumentar en favor de esta tesis? ¿Cómo se denomina su posición?

\vspace{2cm}

\noindent\textbf{4.} ¿Qué tipo de teoría es adecuada como teoría del significado según Davidson?

\vspace{2cm}

\end{document}

%%% Local Variables:
%%% mode: LaTeX
%%% TeX-master: t
%%% End:
